\documentclass{article}
\usepackage{ctex}   

\begin{document}

\title{强化学习方法综述}
\author{张云涛}
\date{\today}
\maketitle

\begin{abstract}
\textbf{摘要:} 

\textbf{关键字:} 强化学习 
\end{abstract}


马尔可夫决策
是对完全可观测到环境进行描述的,
几乎所有的强化学习问题都可以转化为MDP。
当前状态决定未来。

马尔可夫决策过程
在20世纪50年代有美国数学家理查德·贝尔曼提出;
60年代由罗纳德·霍华德在著作《动态规划与卡尔科夫过程》中建立成熟的数学模型,自此广泛应用于
经济学、电子通讯、工程学、社会学等各领域;
90年代以来,用于人工智能领域的研究,
英国剑桥学者沃特金斯在博士论文中基于马尔可夫决策过程提出Q-Learning,为强化学习提供了理论依据。


发展分析
瓶颈
求解最优策略组合的过程实际是一个求解以状态S为参数的多项式方程的过程,而在模拟马尔可夫决策过程
时状态s的数量会以指数级增长,目前可操作的计算量在千万级左右。

未来发展方向
马尔可夫决策过程作为强化学习的基本架构,随着强化学习在AI领域的发展,相信日后还会有进一步的发展。



经典任务使用强化学习的优势
只要RL策略网络收敛了,实时性就比较好,
实时地根据当前情况调整机械臂运动策略(传统方法可以做到吗[笑哭])。
而且RL的model-free方法不需要对机械臂建模,省去建模,解正逆运动学的麻烦。





Sutton的经典书籍 cite

SARSA = 
马尔可夫决策

蒙特卡洛方法

动态规划

Q-Learning



\end{document}
